%!TEX root = ../main.tex

\appendix

\chapter{Rust implementation of inference algorithm}
\label{app:inference-code}

\rustfile[firstline=1,lastline=18]{code/shape_inference/src/lib.rs}
\newpage
\rustfile[firstline=20,lastline=86]{code/shape_inference/src/lib.rs}
\newpage
\rustfile[firstline=88,lastline=110]{code/shape_inference/src/lib.rs}

\chapter{Naive implementation of generation of Rust types from shapes}
\label{app:codegen-code}

\rustfile[firstline=113,lastline=124]{code/shape_inference/src/lib.rs}
\newpage
\rustfile[firstline=125]{code/shape_inference/src/lib.rs}

%\usepackage[includeheadfoot, inner=3.5cm, outer=3.5cm]{geometry}
\newgeometry{inner=3cm, outer=2.5cm, top=4cm, bottom=3cm}
\fancyhfoffset[E,O]{0pt}

\chapter{Examples of generated Rust types}
\label{app:examples-generated-types}

This appendix contains examples of the result of using the code generation of «json_typegen» on various real-world JSON samples.

The JSON shown may be abridged, where such changes do not affect the inference results. E.g. an array with 50 elements may have been reduced to only 2 or 3.

\section{Launch Library Launch List}
\label{app:launchlibrary}

API documentation: \url{https://launchlibrary.net/1.2/docs/api.html} \\
Sample source: \url{https://launchlibrary.net/1.2/launch?next=2}

\begin{multicols}{2}

\subsubsection{JSON Sample}

\begin{minted}[breaklines, fontsize=\scriptsize]{json}
{
  "total": 2,
  "launches": [
    {
      "id": 1329,
      "name": "Vega | OptSat 3000 & VENµS (VENUS)",
      "net": "August 2, 2017 01:58:00 UTC",
      "tbdtime": 0,
      "tbddate": 0
    },
    {
      "id": 1233,
      "name": "Long March 3B/E | Alcomsat-1",
      "net": "August 5, 2017 00:00:00 UTC",
      "tbdtime": 1,
      "tbddate": 1
    }
  ],
  "offset": 0,
  "count": 2
}
\end{minted}
%\end{multicols}

\subsubsection{Generated code}

%\begin{multicols}{2}
\begin{minted}[breaklines, fontsize=\scriptsize]{rust}
#[derive(Default, Debug, Clone, PartialEq, Serialize, Deserialize)]
struct LaunchList {
    total: i64,
    launches: Vec<Launch>,
    offset: i64,
    count: i64,
}

#[derive(Default, Debug, Clone, PartialEq, Serialize, Deserialize)]
struct Launch {
    id: i64,
    name: String,
    net: String,
    tbdtime: i64,
    tbddate: i64,
}
\end{minted}
\end{multicols}

\newpage
\section{CrossRef DOI API}
\label{app:crossref}

API documentation: \url{https://github.com/CrossRef/rest-api-doc} \\
Sample source: \url{https://api.crossref.org/works/10.1145/2908080.2908115/}

\subsubsection{JSON Sample}

\begin{multicols}{2}
\begin{minted}[breaklines, fontsize=\scriptsize]{json}
{
  "status": "ok",
  "message-type": "work",
  "message-version": "1.0.0",
  "message": {
    "indexed": { "date-parts": [[2017, 7, 25]], "date-time": "2017-07-25T04:31:32Z", "timestamp": 1500957092169 },
    "publisher-location": "New York, New York, USA",
    "reference-count": 26,
    "publisher": "ACM Press",
    "license": [
      {
        "URL": "http://www.acm.org/publications/policies/copyright_policy#Background",
        "start": { "date-parts": [[2016, 6, 13]], "date-time": "2016-06-13T00:00:00Z", "timestamp": 1465776000000 },
        "delay-in-days": 164,
        "content-version": "vor"
      }
    ],
    "content-domain": { "domain": [ ], "crossmark-restriction": false },
    "short-container-title": [ ],
    "published-print": { "date-parts": [[2016]] },
    "DOI": "10.1145/2908080.2908115",
    "type": "proceedings-article",
    "created": { "date-parts": [[2016, 6, 2]], "date-time": "2016-06-02T19:23:42Z", "timestamp": 1464895422000 },
    "source": "Crossref",
    "is-referenced-by-count": 0,
    "title": ["Types from data: making structured data first-class citizens in F#"],
    "prefix": "10.1145",
    "author": [
      { "given": "Tomas", "family": "Petricek", "affiliation": [{ "name": "University of Cambridge, UK" }] },
      { "given": "Gustavo", "family": "Guerra", "affiliation": [{ "name": "Microsoft, UK" }] },
      { "given": "Don", "family": "Syme", "affiliation": [{ "name": "Microsoft Research, UK" }] }
    ],
    "member": "320",
    "reference": [
      {
        "key": "key-10.1145/2908080.2908115-1",
        "unstructured": "L. Cardelli and J. C. Mitchell. Operations on Records. In Mathematical Foundations of Programming Semantics, pages 22&#8211;52. Springer, 1990.",
        "DOI": "10.1007/BFb0040253",
        "doi-asserted-by": "crossref"
      },
      {
        "key": "key-10.1145/2908080.2908115-2",
        "unstructured": "A. Chlipala. Ur: Statically-typed Metaprogramming with Type-level Record Computation. In ACM SIGPLAN Notices, volume 45, pages 122&#8211;133. ACM, 2010."
      }
    ],
    "event": {
      "name": "the 37th ACM SIGPLAN Conference",
      "location": "Santa Barbara, CA, USA",
      "sponsor": ["SIGPLAN, ACM Special Interest Group on Programming Languages"],
      "acronym": "PLDI 2016",
      "number": "37",
      "start": { "date-parts": [[2016, 6, 13]] },
      "end": { "date-parts": [[2016, 6, 17]] }
    },
    "container-title": [
      "Proceedings of the 37th ACM SIGPLAN Conference on Programming Language Design and Implementation - PLDI 2016"
    ],
    "original-title": [ ],
    "deposited": { "date-parts": [[2017, 6, 24]], "date-time": "2017-06-24T15:39:00Z", "timestamp": 1498318740000 },
    "score": 1,
    "subtitle": [ ],
    "short-title": [ ],
    "issued": { "date-parts": [[2016]] },
    "ISBN": ["9781450342612"],
    "references-count": 26,
    "URL": "http://dx.doi.org/10.1145/2908080.2908115",
    "relation": { "cites": [ ] }
  }
}
\end{minted}
\end{multicols}

\subsubsection{Generated code}

\begin{multicols}{2}
\begin{minted}[breaklines, fontsize=\scriptsize]{rust}
#[derive(Default, Debug, Clone, PartialEq, Serialize, Deserialize)]
struct CrossRefMetadata {
    status: String,
    #[serde(rename = "message-type")]
    message_type: String,
    #[serde(rename = "message-version")]
    message_version: String,
    message: Message,
}

#[derive(Default, Debug, Clone, PartialEq, Serialize, Deserialize)]
struct Message {
    indexed: Indexed,
    #[serde(rename = "publisher-location")]
    publisher_location: String,
    #[serde(rename = "reference-count")]
    reference_count: i64,
    publisher: String,
    license: Vec<License>,
    #[serde(rename = "content-domain")]
    content_domain: ContentDomain,
    #[serde(rename = "short-container-title")]
    short_container_title: Vec<::serde_json::Value>,
    #[serde(rename = "published-print")]
    published_print: PublishedPrint,
    #[serde(rename = "DOI")]
    doi: String,
    #[serde(rename = "type")]
    type_field: String,
    created: Created,
    source: String,
    #[serde(rename = "is-referenced-by-count")]
    is_referenced_by_count: i64,
    title: Vec<String>,
    prefix: String,
    author: Vec<Author>,
    member: String,
    reference: Vec<Reference>,
    event: Event,
    #[serde(rename = "container-title")]
    container_title: Vec<String>,
    #[serde(rename = "original-title")]
    original_title: Vec<::serde_json::Value>,
    deposited: Deposited,
    score: i64,
    subtitle: Vec<::serde_json::Value>,
    #[serde(rename = "short-title")]
    short_title: Vec<::serde_json::Value>,
    issued: Issued,
    #[serde(rename = "ISBN")]
    isbn: Vec<String>,
    #[serde(rename = "references-count")]
    references_count: i64,
    #[serde(rename = "URL")]
    url: String,
    relation: Relation,
}

#[derive(Default, Debug, Clone, PartialEq, Serialize, Deserialize)]
struct Indexed {
    #[serde(rename = "date-parts")]
    date_parts: Vec<Vec<i64>>,
    #[serde(rename = "date-time")]
    date_time: String,
    timestamp: i64,
}

#[derive(Default, Debug, Clone, PartialEq, Serialize, Deserialize)]
struct License {
    #[serde(rename = "URL")]
    url: String,
    start: Start,
    #[serde(rename = "delay-in-days")]
    delay_in_days: i64,
    #[serde(rename = "content-version")]
    content_version: String,
}

#[derive(Default, Debug, Clone, PartialEq, Serialize, Deserialize)]
struct Start {
    #[serde(rename = "date-parts")]
    date_parts: Vec<Vec<i64>>,
    #[serde(rename = "date-time")]
    date_time: String,
    timestamp: i64,
}

#[derive(Default, Debug, Clone, PartialEq, Serialize, Deserialize)]
struct ContentDomain {
    domain: Vec<::serde_json::Value>,
    #[serde(rename = "crossmark-restriction")]
    crossmark_restriction: bool,
}

#[derive(Default, Debug, Clone, PartialEq, Serialize, Deserialize)]
struct PublishedPrint {
    #[serde(rename = "date-parts")]
    date_parts: Vec<Vec<i64>>,
}

#[derive(Default, Debug, Clone, PartialEq, Serialize, Deserialize)]
struct Created {
    #[serde(rename = "date-parts")]
    date_parts: Vec<Vec<i64>>,
    #[serde(rename = "date-time")]
    date_time: String,
    timestamp: i64,
}

#[derive(Default, Debug, Clone, PartialEq, Serialize, Deserialize)]
struct Author {
    given: String,
    family: String,
    affiliation: Vec<Affiliation>,
}

#[derive(Default, Debug, Clone, PartialEq, Serialize, Deserialize)]
struct Affiliation {
    name: String,
}

#[derive(Default, Debug, Clone, PartialEq, Serialize, Deserialize)]
struct Reference {
    key: String,
    unstructured: String,
    #[serde(rename = "DOI")]
    doi: Option<String>,
    #[serde(rename = "doi-asserted-by")]
    doi_asserted_by: Option<String>,
}

#[derive(Default, Debug, Clone, PartialEq, Serialize, Deserialize)]
struct Event {
    name: String,
    location: String,
    sponsor: Vec<String>,
    acronym: String,
    number: String,
    start: Start2,
    end: End,
}

#[derive(Default, Debug, Clone, PartialEq, Serialize, Deserialize)]
struct Start2 {
    #[serde(rename = "date-parts")]
    date_parts: Vec<Vec<i64>>,
}

#[derive(Default, Debug, Clone, PartialEq, Serialize, Deserialize)]
struct End {
    #[serde(rename = "date-parts")]
    date_parts: Vec<Vec<i64>>,
}

#[derive(Default, Debug, Clone, PartialEq, Serialize, Deserialize)]
struct Deposited {
    #[serde(rename = "date-parts")]
    date_parts: Vec<Vec<i64>>,
    #[serde(rename = "date-time")]
    date_time: String,
    timestamp: i64,
}

#[derive(Default, Debug, Clone, PartialEq, Serialize, Deserialize)]
struct Issued {
    #[serde(rename = "date-parts")]
    date_parts: Vec<Vec<i64>>,
}

#[derive(Default, Debug, Clone, PartialEq, Serialize, Deserialize)]
struct Relation {
    cites: Vec<::serde_json::Value>,
}
\end{minted}
\end{multicols}

\newpage
\section{Steam Store News}

API documentation: \url{https://developer.valvesoftware.com/wiki/Steam_Web_API} \\
Sample source: \url{http://api.steampowered.com/ISteamNews/GetNewsForApp/v0002/?appid=252950&count=2&maxlength=150}

\subsubsection{JSON Sample}

\begin{multicols}{2}
\begin{minted}[breaklines, fontsize=\scriptsize]{json}
{
  "appnews": {
    "appid": 252950,
    "newsitems": [
      {
        "gid": "2079952210065770645",
        "title": "DreamHack Atlanta Rocket League Championship Preview",
        "url": "http://store.steampowered.com/news/externalpost/steam_community_announcements/2079952210065770645",
        "is_external_url": true,
        "author": "Dirkened",
        "contents": "https://rocketleague.media.zestyio.com/DreamHack-Atlanta.c6e1dc555a6eff57c623d9877706c9a5.jpg With the conclusion of the FACEIT X Games Rocket League ...",
        "feedlabel": "Community Announcements",
        "date": 1500498351,
        "feedname": "steam_community_announcements",
        "feed_type": 1,
        "appid": 252950
      },
      {
        "gid": "2472890725411073707",
        "title": "RLCS Season 4 Kicks Off this August",
        "url": "http://store.steampowered.com/news/externalpost/steam_community_announcements/2472890725411073707",
        "is_external_url": true,
        "author": "Dirkened",
        "contents": "https://rocketleague.media.zestyio.com/rlcs_screen--1-.c6e1dc555a6eff57c623d9877706c9a5.png Just six weeks ago, we were celebrating the Rocket League ...",
        "feedlabel": "Community Announcements",
        "date": 1500313284,
        "feedname": "steam_community_announcements",
        "feed_type": 1,
        "appid": 252950
      }
    ],
    "count": 341
  }
}
\end{minted}
\end{multicols}

\subsubsection{Generated code}

\begin{multicols}{2}
\begin{minted}[breaklines, fontsize=\scriptsize]{rust}
#[derive(Default, Debug, Clone, PartialEq, Serialize, Deserialize)]
struct SteamAppNews {
    appnews: Appnews,
}

#[derive(Default, Debug, Clone, PartialEq, Serialize, Deserialize)]
struct Appnews {
    appid: i64,
    newsitems: Vec<Newsitem>,
    count: i64,
}

#[derive(Default, Debug, Clone, PartialEq, Serialize, Deserialize)]
struct Newsitem {
    gid: String,
    title: String,
    url: String,
    is_external_url: bool,
    author: String,
    contents: String,
    feedlabel: String,
    date: i64,
    feedname: String,
    feed_type: i64,
    appid: i64,
}
\end{minted}
\end{multicols}

\newpage
\section{World Bank Indicator}
\label{app:worldbank}

API documentation: \url{https://datahelpdesk.worldbank.org/knowledgebase/topics/125589} \\
Sample source: \url{http://api.worldbank.org/countries/no/indicators/NY.GDP.MKTP.CD?format=json}

\begin{multicols}{2}

\subsubsection{JSON Sample}

\begin{minted}[breaklines, fontsize=\scriptsize]{json}
[
  { "page": 1, "pages": 2, "per_page": "50", "total": 57 },
  [
    {
      "indicator": {
        "id": "NY.GDP.MKTP.CD",
        "value": "GDP (current US$)"
      },
      "country": { "id": "NO", "value": "Norway" },
      "value": "386383919342.271",
      "decimal": "0",
      "date": "2009"
    },
    {
      "indicator": {
        "id": "NY.GDP.MKTP.CD",
        "value": "GDP (current US$)"
      },
      "country": { "id": "NO", "value": "Norway" },
      "value": "461946808510.638",
      "decimal": "0",
      "date": "2008"
    },
    {
      "indicator": {
        "id": "NY.GDP.MKTP.CD",
        "value": "GDP (current US$)"
      },
      "country": { "id": "NO", "value": "Norway" },
      "value": "400883873279.083",
      "decimal": "0",
      "date": "2007"
    }
  ]
]
\end{minted}
%\end{multicols}

\subsubsection{Generated code}

%\begin{multicols}{2}
\begin{minted}[breaklines, fontsize=\scriptsize]{rust}
#[derive(Default, Debug, Clone, PartialEq, Serialize, Deserialize)]
struct WorldBankIndicator {
    page: i64,
    pages: i64,
    per_page: String,
    total: i64,
}

#[derive(Default, Debug, Clone, PartialEq, Serialize, Deserialize)]
struct WorldBankIndicator2 {
    indicator: Indicator,
    country: Country,
    value: String,
    decimal: String,
    date: String,
}

#[derive(Default, Debug, Clone, PartialEq, Serialize, Deserialize)]
struct Indicator {
    id: String,
    value: String,
}

#[derive(Default, Debug, Clone, PartialEq, Serialize, Deserialize)]
struct Country {
    id: String,
    value: String,
}
\end{minted}
\end{multicols}


\restoregeometry{}
\fancyhfoffset[E,O]{0pt}

\chapter{Comparison of project setup with dependencies in Rust and C++}
\label{app:cargo-cpp-comparison}

\subsubsection{Minimal project setup in Rust}

\begin{enumerate}
  \item Install the Rust toolchain:\\
        ‹curl https://sh.rustup.rs -sSf | sh›
  \item Create a new (binary) project:\\
        ‹cargo new --bin timer && cd timer›
  \item Add a dependency to the Cargo configuration file:\\
        ‹echo 'tokio-timer = "*"' >Cargo.toml›
  \item Write some actual Rust code:\\
        \icode{{\$}EDITOR src/main.rs}
  \item Build and run:\\
        ‹cargo run›
\end{enumerate}

% curl https://sh.rustup.rs -sSf | sh
% cargo new --bin timer && cd timer
% echo 'tokio-timer = "*"' >Cargo.toml
% $EDITOR src/main.rs
% cargo run

\newpage

\subsubsection{Minimal project setup in \cpp}

\begin{enumerate}
  \item Choose and install a \cpp\ toolchain. What toolchains are available and how to install them differs from platform to platform, and some may have a toolchain installed. To keep things somewhat simple, installation is omitted here.
  \item Choose and install a dependency manager. Throughout the years there have been several \cpp\ dependency manager projects that have come and gone, so this choice is not entirely without risks. Without going into the details why, I chose «conan»\footnote{\url{http://conan.io}}. The rest of these steps are based on the «conan» \say{Getting started}-guide\footnote{\url{http://docs.conan.io/en/latest/getting_started.html}}. Again installation depends somewhat on platform. For my part I installed it using «homebrew»\footnote{Which itself would have to been installed if it was not so already.}:\\
        ‹brew install conan›
  \item Choose and install a build tool. Since the «conan» guide uses «CMake» this is also what we will use here. Again, installation method differs, but for my part:\\
        ‹brew install cmake›
  \item Make a project directory:\\
        ‹mkdir timer && cd timer›
  \item Make and write a «conan» configuration file:\\
        \icode{{\$}EDITOR conanfile.txt}
  \item Write some actual \cpp\ code:\\
        \icode{{\$}EDITOR timer.cpp}
  \item Make and write a «CMake» configuration file:\\
        \icode{{\$}EDITOR CMakeLists.txt}
  \item Make a build directory:\\
        ‹mkdir build && cd build›
  \item Install dependencies:\\
        ‹conan install ..›
  \item Prepare build:\\
        ‹cmake ..›
  \item Build:\\
        ‹cmake --build .›
  \item Run the resulting program:\\
        ‹bin/timer›
\end{enumerate}

% Install a \cpp\ toolchain
% brew install conan
% brew install cmake
% mkdir timer && cd timer
% $EDITOR conanfile.txt
% $EDITOR timer.cpp
% $EDITOR CMakeLists.txt
% mkdir build && cd build
% conan install ..
% cmake ..
% cmake --build .
% bin/timer
