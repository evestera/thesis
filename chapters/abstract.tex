%!TEX root = ../main.tex

\vspace*{2cm}
\thispagestyle{plain}

\begin{center}

\phantomsection
\addcontentsline{toc}{section}{Abstract}

\section*{\hfil Abstract \hfil}

\end{center}

When programmers access external data in a statically typed programming language, they are often faced with a dilemma between convenient and type-safe access to the data.

In the programming language F\#, a concept called type providers has been proposed as a solution to this problem by having compiler support for libraries with the capability to generate types at compile time.

This thesis presents «json_typegen», a project which aims to show the feasibility of similar solutions in the Rust programming language. The project uses compile-time metaprogramming along with alternative interfaces to the same code generation implementation to achieve convenient, type-safe access to data in the JSON data format. While JSON is chosen as the format for the presented library, the approach also applies to other data formats and sources.
